\documentclass[12pt]{article}
\usepackage{polski}
\usepackage[utf8]{inputenc}
\usepackage[T1]{fontenc}
\usepackage{times}

\usepackage{amsfonts}
\usepackage{amsmath}
\usepackage{bm}
\usepackage{mathtools}
\mathtoolsset{showonlyrefs}


\usepackage{tabularx}
\usepackage{array}
\newcolumntype{Y}{>{\centering\arraybackslash}X}
\newcolumntype{Z}{>{\centering\arraybackslash}p}
\usepackage{multirow}
\usepackage{hyperref}

\usepackage{enumitem}
\usepackage{float}


\usepackage{graphicx}
\usepackage{rotating}
\usepackage{subcaption}


%\usepackage{animate}

\renewcommand{\thesection}{\arabic{section}}
\renewcommand{\thesubsection}{\arabic{section}.\arabic{subsection}}
\usepackage{wrapfig}



\usepackage{amsmath}
\usepackage{amsthm}
\usepackage{dsfont}
\newtheorem{lema}{Lemma}

\newtheoremstyle{exer}{20pt}{10pt}{}{0pt}{\bfseries}{.\\}{.5em}{}
\theoremstyle{exer}
\newtheorem{ex}{Ex}

%\usepackage[a4paper,total={6in,10in}]{geometry}
\usepackage[a4paper,total={7in,10in}]{geometry}
%\newtheoremstyle{style name}{space above}{space below}{body font}{indent amount}{head font}{head punct}{after head space}{head spec}


\begin{document}
	\begin{titlepage}
		\begin{center}
			
			\textbf{\Huge  Analiza danych rzeczywistych przy pomocy modelu ARMA}
			
			\vspace{0.5cm}
			
			\vspace{1.5cm}
			
			\textbf{\LARGE Autorzy}\\
			\vspace{0.5cm}
			\large Kacper Budnik, 262286\\
			\large Maciej Karczewski, 262282\\
			
			
			\vfill
			
			\vspace{0.4cm}
			

			
			\vspace{0.8cm}
			Wydział Matematyki	
			\today
		\end{center}
	\end{titlepage}
	\tableofcontents
	\newpage
	
	\section{Wprowadzenie}
	
	\section{Przygotowanie danych do analizy}
	
	
	\section{Dekompozycja szeregu czasowego}
	\subsection{Wykresy dla surowych danych}
	
	\subsection{Transformacja Boxa-Coxa}

	\subsection{Różnicowanie sezonowe}
	
	
	

	
	\section{Analiza residuów}
	Podczas tworzenia modelu regresji liniowej oraz dalszych obliczeń zakładaliśmy następujące warunki
	\begin{enumerate}
		\item $\mathbb{E}\xi_i=0$ $\forall i$ ,
		\item $Var\xi_i=\sigma^2<\infty\quad\forall i$,
		\item $\xi_i$ mają rozkład normalny,
		\item $\xi_i\perp\!\!\!\perp\xi_j$ dla $i\neq j$.
	\end{enumerate}

	\section{Wnioski autorów}
	
	
\end{document}